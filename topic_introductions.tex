\documentclass[11pt]{amsart}
\usepackage{geometry}                % See geometry.pdf to learn the layout options. There are lots.
\geometry{letterpaper}                   % ... or a4paper or a5paper or ... 
%\geometry{landscape}                % Activate for for rotated page geometry
%\usepackage[parfill]{parskip}    % Activate to begin paragraphs with an empty line rather than an indent
\usepackage{graphicx}
\usepackage{amssymb}
\usepackage{epstopdf}
\DeclareGraphicsRule{.tif}{png}{.png}{`convert #1 `dirname #1`/`basename #1 .tif`.png}

\title{Outlining Weekly Topics of Focus for Summer 2021 Research Experience}
\author{Daniel Muratore}
%\date{}                                           % Activate to display a given date or no date

\begin{document}
\maketitle
\section{Introduction}
\paragraph{} The purpose of this document is to act as a guide and de facto `syllabus' for your research rotation with our group. I'm going to outline a general plan for every week that will include readings as well as some practice project. By no means is this plan set in stone, nor are we obligated to stick to it. If you would like to adjust the pace or material, please tell me and we can do that. My goal for this rotation is to give you a thorough introduction to our field of research, but primarily to facilitate you developing the skills and expertise you want to develop. To that end, if you want to spend more or less time focusing on a particular topic, then we will. I am planning to keep this document updating in real time via a github repository that we'll work together on using, so we both can keep up with changes. While we will start discussing your final project fairly shortly after beginning to work together, we will spend a couple of weeks going over background and topics in marine viral ecology to help inform a good choice of final project, which we'll design and start working on in earnest a couple of weeks in. 
\subsection{Readings}
\paragraph{} Reading and writing are always great skills to practice. Every week we'll be covering some important classic/foundational papers in marine viral ecology and discussing them together. Hopefully this will give you a thorough introduction to our field, and I am excited to revisit and gain new perspective from hearing your interpretations of these papers. Apart from papers important for background, we will also be engaging with recent literature from the past year or so to get a good idea for what's hot in the field right now and where the boundaries of knowledge are currently being pushed. Please tell me if the reading load is overbearing, or if you've decided to do some independent reading and would like to show me a paper you found! I will post pdfs of the papers I selected on the github, so there should be no problems with respect to paywalls, but should you encounter a paywall please tell me and I will gladly hunt down the paper for you. Because reading and writing go together, I want us to try a new mini-project. Every week we will spend a little time writing a small summary of a recent paper (maybe 250 words or so), which will then be published on the Weitz Group lab website. My hope is that the exercises is 1) fun, 2) deepens our understanding of the recent literature, 3) provides a helpful introduction to other scientists who do not want to read the entire papers, and 4) leaves an archived record of a science communication project that can go on your resume. If those goals don't end up being met, we can drop the idea. 
\subsection{Practices} The practice is meant to give you a chance to apply the ideas we discuss in the readings and develop your coding skills. Partway through your rotation, we will move on to deciding your project and phase out the practices for project progress goals. Please ask lots of questions about the practices, I will gladly help if you are feeling stuck or frustrated with a coding bug (it happens to the best of us). Do not stress about creating a fancy report or presentation. Once you complete a practice or have gotten everything you need out of working on a practice, we'll discuss your results together. For the practices, which will all have coding components, please use any programming language you like, though I will be able to help you the most with the technical details if you work in R, Python, Matlab, or Julia. The only thing is I will ask you to push your code to this github repo to give you practice with using github and also so you have a public archive of coding projects you've done that can go on your resume. 
\subsection{Project} 
\paragraph{} I will fill out this section more once we discuss expectations and timelines for the project together. Generally, over the first couple of weeks focusing on background we'll identify what topics in marine viral ecology are most interesting to you and pick a project along those lines. I have a couple of general ideas that I'll present as a starting place. We'll spend several days focusing on outlining appropriate milestones and estimating a timeline for how long those milestones will take. 
\section{Week 1 (May 17-21): Introduction to Viral Ecology Theory}
\subsection{Readings} Jover et al 2014, QVE Chapter 1 selections, Grome and Isaacs 2021
\subsection{Practice} Complete practice in practice week 1 folder in this repository. 
\subsection{REU Milestones} May 17, welcome and orientation meetings. May 17-18 QBioS workshop. May 20 virtual campus tour. May 21 some kind of social event TBD(?)
\section{Week 2 (May 24-28): Introduction to Marine Microbial Ecology and Biological Oceanography}
\subsection{Readings} Karl 1999 review, Zimmerman et al 2020, Ustick et al 2021
\subsection{Practice} Read accompanying text and complete the ecological dynamics lab from Joshua's new textbook (Daniel will personally share the materials since the book isn't published yet). 
\subsection{REU Milestones} May 25 RCR part I, May 27 CoS meet and greet and research plan elevator pitch.
\section{Week 3 (May 31-June 4): Dynamical Systems in Marine Viral Ecology}
\subsection{Readings} Weitz et al 2015, Pourtois et al 2020, Behrenfield et al 2021
\subsection{Practice} Practice problem on NPVZD models (I will post)
\subsection{REU Milestones} June 1 Hypothesis testing and experimental design lecture, June 3 RCR part II
\subsection{Special} May 31 is memorial day. 
\section{Week 4 (June 7-11): Bioinformatics and Data Science in Environmental Microbiology + Writing a Project Proposal/Plan}
\subsection{Readings} Coenen et al 2020, McLaren et al 2019, Cael et al 2018, Jeganathan and Holmes 2021
\subsection{Practice} Work through the tutorial examples provided in Coenen et al 2020 for microbiome analysis. Writing research proposal/plan.
\subsection{REU Milestones} June 8 Scientific Writing part I, June 11 research plan due to me and Joshua. 
\section{Week 5 (June 14-18): Project focus, continue on Bioinformatics and Data Science in Environmental Microbiology}
\subsection{Readings} Continue readings from week 4
\subsection{Project} Start working on project and finish working through tutorial examples provided in Coenen et al 2020. 
\section{Week 6 (June 21-25): Project focus and Topics in Marine Viral Ecology TBD}
\subsection{Readings}
\subsection{Project}
\subsection{Special} Aquatic Virus Workshop being held virtually June 27-July 1, Daniel will figure out how to get you in to those meetings. 
\subsection{REU Milestones} June 22 How to give an effective oral presentation
\section{Week 7 (June 28-July 2): Project focus and Topics in Marine Viral Ecology TBD}
\subsection{Readings}
\subsection{Project}
\subsection{REU Milestones} June 29 Scientific writing part II
\subsection{Special} Aquatic Virus Workshop being held virtually June 27-July 1, Daniel will figure out how to get you in to those meetings. 
\section{Week 8 (July 5-July 9): Project focus and Topics in Marine Viral Ecology TBD (Daniel leaves for pre-fieldwork quarantine)}
\subsection{Readings}
\subsection{Project}
\subsection{Special} Daniel leaves to begin pre-fieldwork quarantine on July 6. After that travel day I'll be completely available remotely (albeit at a -6hr time difference) to check in and talk.
\section{Week 9 (July 12-16): Wrap Up and Project Presentation}
\subsection{Readings}
\subsection{Project}
\subsection{REU Milestones} Practice talks and final oral presentation - Daniel will really try to attend remotely. 
\section{Week 10 (July19-23): Wrap Up (Daniel at sea)}
\subsection{Readings}
\subsection{REU Milestones} July 20 capstone poster session, July 22 final paper due
\subsection{Special} Daniel will be going out to sea with no internet on July 20, so we'll conclude a few days early. 


\end{document}